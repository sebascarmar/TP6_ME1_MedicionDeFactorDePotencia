\subsection{Potencias en un sistema alimentado con corriente alterna}

   Se puede definir el concepto de \textbf{potencia}, como cantidad de energía eléctrica 
  entregada o absorbida por un elemento en un momento determinado, cuya unidad en el Sistema 
  Internacional de Unidades es el \textbf{watt} (\textbf{W}).

   En circuitos de corriente alterna (CA), se puede determinar 3 tipos de potencias:
   \textbf{activa}, \textbf{reactiva} y \textbf{aparente}.

  La potencia activa (\textbf{P}) también conocida como la \textit{potencia real}, es la que
  es consumida por cargas resistivas propias de un circuito, y es medida 
  en \textbf{watt} (\textbf{W}). La potencia reactiva (\textbf{Q}), es la potencia 
  intercambiada por cargas inductivas y/o capacitivas; su unidad de medida es 
  \textbf{volt-ampere reactivo} (\textbf{VAR}). Por último, la potencia aparente (\textbf{S}),
  es una combinación entre la potencia activa y la reactiva; su unidad de medida es 
  \textbf{volt-ampere } (\textbf{VA}).

  Estas potencias se relacionan entre sí de la siguiente manera. Partiendo de una señal

        \vspace{-15pt}  
        \begin{align}
            v(t)   &= V_m \sin(\omega t)\ [V] ~, \notag \\
            \intertext{que excita un circuito genérico que contiene una impedancia 
                        $Z = R + jX = |Z| \angle \varphi$,  se produce una corriente
                        eléctrica cuya forma es}
            i(t)   &= I_m  \sin(\omega t - \varphi)\ [A] \ , \notag \\
            \intertext{entonces, la potencia instantánea, asociada a la señal de 
                excitación, se obtiene mediante el producto de la tensión por la 
                corriente, dando como resultado}
            p(t)  &= [V_m \sin(\omega t)] 
            \cdot [I_m \sin(\omega t - \varphi)]\ [W]  ~.
                            \label{eqn:Winst}
         \end{align}

      
    Ahora, partiendo de la ecuación~(\ref{eqn:Winst}), la misma se la puede reescribir de la
    siguiente forma (mediante la identidad trigonométrica del \textit{seno de la resta})

        \vspace{-15pt}
         \begin{align}
            p(t)  &= V_m \sin(\omega t)\cdot [I_m \sin(\omega t) \cos(\varphi) - I_m \cos(\omega t) \sen(\varphi)]  \notag \\
            p(t)   &= [2 \sin^{2}(\omega t)]\cdot V I \cos(\varphi) - [2 \sin(\omega t) \cos(\omega t)]\cdot V I\sen(\varphi) ~.\notag \\
            \intertext{Luego, aplicando las identidades}
                  &2 \sin^{2}(\alpha)  = 1 - \cos(2\alpha) \hspace{20pt} ; \hspace{20pt} \sin(2\alpha) = 2 \sin(\alpha) \cos(\alpha) \ , \notag \\
            \intertext{finalmente queda la siguiente expresión}
            p(t)  &= [1- \cos(2 \omega t)]\cdot V I \cos(\varphi) - [\sin(2 \omega t)]\cdot V I \sin(\varphi) ~. 
         \label{eqn:Pot_ins}
         \end{align}

      De esta última expresión obtenida se observa que el primer término del segundo miembro
      de la ecuación, corresponde a la \textbf{potencia activa P} asociada a lo componentes
      resistivos de un sistema, mientras que el segundo término del mismo, corresponde a la 
      \textbf{potencia reactiva Q}, asociada a los componentes reactivos del mismo.

      Luego, realizando la integración en el período T de la ecuación~(\ref{eqn:Pot_ins}), se obtiene

        \begin{equation}
            \boxed{P   = V I \cos(\alpha)~[W]}~.  \label{eqn:PotActTot}
        \end{equation}
        \begin{equation}
            \boxed{Q   = V I \cos(\alpha)~[VAR]}~. \label{eqn:PotReacTot}
        \end{equation}

      Por último, la \textbf{potencia aparente total} (\textbf{S}), se obtiene  sumando los aportes de
      las potencias activas y reactivas, y como estas mismas se realizan en cuadratura, entonces

        \begin{equation}
            S^{2} = P^{2} + Q^{2} \hspace{20pt} \therefore \hspace{20pt} \boxed{S = \sqrt{P^{2} + Q^{2}}~[VA]} ~. \label{eqn:PotApaTot}       
         \end{equation}


         \subsubsection{Obtención del triángulo de potencia mediante fasores}
            El cálculo de las potencias resistivas y reactivas se puede obtener de manera
            directa mediante el uso de fasores. A continuación, se deducen los cálculos
            necesarios para la obtención.
            Partiendo de una tensión descrita en fasores $\vec{V} = |\vec{V}| \angle \,  0^{\circ} $,
            excitando un circuito con una impedancia $ Z = |Z| \angle \pm \varphi \ $, se
            genera una corriente $\vec{I}  = \vec{|I|} \angle \mp   \varphi $~.

            La potencia activa total, como se vio en la sección anterior, se puede obtener 
            mediante la ecuación~(\ref{eqn:PotActTot}). La misma se puede reescribir en término de 
            fasores como el \textit{producto de sus módulos por el coseno del ángulo entre ellos}
               \begin{align}
                  P = \vec{|V|} \vec{|I|} \cos(\varphi) = V I \cos(\varphi)~. \notag
               \end{align}   

            \noindent La potencia reactiva total, como se vio en la sección anterior, se puede obtener 
            mediante ecuación~(\ref{eqn:PotReacTot}). La misma se puede reescribir en término de 
            fasores como el \textit{producto de sus módulos por el seno del ángulo entre ellos}
               \begin{align}
                  Q = \vec{|V|} \vec{|I|} \sin(\varphi) = V I \sin(\varphi)~. \notag
               \end{align}               
            
             \noindent Por último, la potencia aparente total (S) se obtiene mediante el 
            \textit{producto de los módulos de los fasores tensión y corriente total}  
               \begin{align}
                  S = \vec{|V|} \vec{|I|} = V I~. \notag
               \end{align}
            
               
             Las potencias activa y reactiva pueden ser vistas como, \textit{
            el producto del módulo del fasor tensión y la proyección de la corriente en 
            el eje real e imaginario respectivamente}
               \begin{equation*}
                  P  = V \cdot(I \cos(\varphi)) \hspace{20pt} ;
                  \hspace{20pt} Q  = V \cdot(I \sin(\varphi))~.
               \end{equation*}   

            \noindent A la proyección de la corriente sobre el eje real, se la conoce como 
            \textbf{corriente activa}, y, a su vez, la proyección de la corriente en el eje 
            imaginario se lo conoce como \textbf{corriente reactiva},tal y como se ve en la 
            Figura~\ref{fig:DiagFasor}. 

            Además, las potencias resistivas y reactivas, están relacionadas entre sí de 
            tal forma de que, si se suman los cuadrados de las mismas se obtiene
               \begin{align}
                 P^{2} + Q^{2} &=  {(V I \cos(\varphi))}^2 + {(V I \sin(\varphi))}^2 = {(V I)}^2 \ , \notag \\
                  \intertext{de donde se deduce la potencia aparente}
                     S^{2}\ &=\ P^{2}\ + Q^{2} \hspace{10pt} \rightarrow \hspace{10pt} S = \sqrt{P^{2} + Q^{2}} \ . \notag
               \end{align}
            
            Debido a esta relación entre las 3 potencias, se conforma lo que se conoce como 
            \textbf{triángulo de potencia} como se ve en la Figura~\ref{fig:TriangPot}.

               \begin{figure}[H]
                  \centering
                     \begin{subfigure}[b]{0.4\textwidth}
                        \centering
                        \frame{\includegraphics[width=\textwidth]{Imagenes/MarcoTeorico/DiagramaFasorDePotencias.png}}
                        \caption{Diagrama Fasorial.}
                        \label{fig:DiagFasor}
                  \end{subfigure}
                  \hfill
                  \begin{subfigure}[b]{0.5\textwidth}
                        \centering
                        \frame{\includegraphics[width=\textwidth]{Imagenes/MarcoTeorico/TrianguloDePotencias.png}}
                        \caption{Triángulo de potencias.}
                        \label{fig:TriangPot}
                  \end{subfigure}
                  \hfill
                  \caption{Obtención del triangulo de potencias.}
                  \label{fig:ObtDeTriagDePot}
               \end{figure}
          
         \subsubsection{Factor de potencia}
         
         El \textbf{factor de potencia} (\textbf{fp}), se lo puede definir como la relación que
               existe entre la energía que es aprovechada, capaz de realizar un trabajo, y la 
               energía total del sistema disponible.
               Dichas energías se relacionan, directamente con la potencia activa (P) y 
               la potencia aparente (S) 
               \begin{equation}
               fp = \frac{W_p}{W_s} = \frac{P}{S} = \dfrac{V I \cos(\varphi)}{V I} \notag \\
                        \hspace{20pt} \therefore \hspace{20pt} \boxed{fp = \cos(\varphi)}~.\label{eqn:fpFinal} 
               \end{equation}

               El factor de potencia, es un valor adimensional que indica el 
               \textbf{rendimiento de un sistema}, es decir, qué parte de la energía disponible 
               se transforma en trabajo. Dicho valor es un número comprendido entre 0 y 1, donde 
               0 indica que un sistema es\textbf{puramente reactivo}, lo cual quiere decir que 
               toda la energía disponible en él retorna a la fuente en cada ciclo. En el caso de
                un $fp = 1$ el sistema es \textbf{puramente resistivo}, es decir, toda la energía
               disponible en él es consumida por una carga transformándose en trabajo.

               En la práctica, se busca un fp lo más cercano a 1 para aprovechar al máximo la 
               energía. En caso de no tener un valor cercano a la unidad, se realiza lo que se 
               conoce como \textit{corrección de factor de potencia}, el cual se explica en la 
               siguiente sección.
               El factor de potencia, al igual que las potencias reactivas, es expresado en 
               términos de \textit{adelanto}, para sistemas de carácter inductivo, o 
               \textit{atraso de fase}, para sistemas de carácter capacitivo.

            \subsubsection{Corrección del factor de potencia}
               
               Como se mencionó anteriormente, en los sistemas se busca un factor de potencia
               lo más cercano a 1, debido a que \textbf{la energía que es transportada 
               y no se consume produce pérdidas}. Por ende, entre más cercano sea el valor a la unidad, las pérdidas
               serán mínimas en el sistema, pero en caso de ser más alejado al valor unitario, se
               producirán mayores pérdidas en el mismo, debiendo realizar una 
               \textbf{corrección del factor de potencia}.

               Dicha corrección se logra, \textbf{conectando cargas reactivas en paralelo, 
               de carácter contrario al que posee el sistema}, de ésta manera no se modifica la tensión aplica-
               da en el mismo. El cálculo de corrección es realizado en base al factor de potencia
               deseado. Para la ejecución del trabajo práctico, se tiene en cuenta la corrección del fp
               para una carga inductiva mediante el siguiente planteamiento

               Se parte de un factor de potencia inicial \(fp_0\) a un factor 
               de potencia final \(fp_f\)
               \begin{align}
                  fp_0  &= \cos(\varphi_0) \hspace{10pt} y \hspace{10pt} fp_f = \cos(\varphi_f)~,  \notag
               \end{align}

               \noindent y considerando ambos en atraso, para compensar el sistema se 
               conecta una carga capacitiva de potencia reactiva \(Q_C\)
               \begin{align}  
                  Q_f   &= Q_0 - Q_C \hspace{10pt}\Longrightarrow\hspace{10pt} Q_C = Q_0 - Q_f ~. \notag
               \end{align}

               \noindent Reemplazando en la expresión anterior la ecuación (\ref{eqn:PotReacTot}) para 
               cada potencia reactiva
               \begin{align}   
                  Q_C   &= V I_0 \sin(\varphi_0) - V I_f \sin(\varphi_f)  \notag \\ 
                  Q_C   &= V I_0 \sin(\varphi_0)\cdot\dfrac{\cos(\varphi_0)}{\cos(\varphi_0)}
                         - V I_f \sin(\varphi_f)\cdot\dfrac{\cos(\varphi_f)}{\cos(\varphi_f)}  \notag \\ 
                  Q_C   &= V I_0 \cos(\varphi_0)\cdot \tan(\varphi_0)  - V I_f \cos(\varphi_f)\cdot \tan(\varphi_f) \ , \notag
               \end{align}   

               \noindent como la potencia activa P no varía entonces
               \begin{align}
                  Q_C   &= P\cdot (\tan(\varphi_0) - \tan(\varphi_f))~. \notag
               \end{align}

               \noindent Finalmente, teniendo en cuenta que $Q_C= \dfrac{V^{2}}{X_C}$, y la expresión anterior, se igualan
               y se obtiene 
               \begin{align}
                  \dfrac{V^{2}}{X_C} &=  P\cdot (\tan(\varphi_0) - \tan(\varphi_f))  \notag\\
                  V^{2}\cdot \omega C &= P\cdot (\tan(\varphi_0) - \tan(\varphi_f))  \notag\\[3mm]
                  \Longrightarrow \hspace{20pt}C &= \dfrac{P\cdot (\tan(\varphi_0) - \tan(\varphi_f))}{V^{2}\cdot \omega}~. \notag
               \end{align}

               \noindent La ecuación antes mencionada, hace referencia a la obtención de una capacitancia 
               para un valor de \(fp_f\) como referencia. Para el desarrollo del presente trabajo práctico, 
               se tendrá en consideración un \(fp_f = 1\), para obtener la capacitancia buscada.
               De esta manera la ecuación final queda reducida a la siguiente expresión
                  \begin{equation}
                    \boxed{C = \dfrac{P\cdot \tan(\varphi_0)}{V^{2}\cdot \omega}~~[F]}~. \label{eqn:CorrecFp}
                  \end{equation}

               
               



							












